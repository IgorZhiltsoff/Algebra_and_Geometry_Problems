%\documentclass[a4paper, 12pt]{article}


%%% Template
\newcommand{\coursename}{Алгем (доп. семинар)}
\newcommand{\coursedate}{UNDEFINED}
\newcommand{\term}{UNDEFINED}


%%% Lang support
\usepackage[T2A]{fontenc}
\usepackage[utf8]{inputenc}
\usepackage[russian,english]{babel}


%%% Environments
\usepackage{amsthm}

% Problem
\theoremstyle{definition}
\newtheorem{problem}{}

% Definition
\theoremstyle{definition}
\newtheorem*{definition}{О}

%%% Operations
\newcommand{\tr}{\mathbf{tr}}
\newcommand{\detbf}{\mathbf{det}}

%%% Page format
% space outline
\usepackage{geometry}
\geometry{top=25mm}
\geometry{left=15mm}
\geometry{right=15mm}
\geometry{bottom=10mm}

% indentation
\usepackage{indentfirst}

% header
\usepackage{titleps}
\newpagestyle{main}{
	\setheadrule{0.4pt}
	\sethead{\coursename, семестр \term}{}{ФПМИ МФТИ, \coursedate}
}
\pagestyle{main}

% heading commands
\newcommand{\heading}[1]{\Large \centerline{\textbf{#1}}}
\newcommand{\subheading}[1]{\large \centerline{\textbf{#1}}}

% line splitting (use \seqsplit)
\usepackage{seqsplit}


%%% Lists
% mix inline and breakline in lists 
% Borrowed from: https://tex.stackexchange.com/a/51089/222164
\usepackage[inline]{enumitem}   
\makeatletter
% This command ignores the optional argument for itemize and enumerate lists
\newcommand{\inlineitem}[1][]{%
	\ifnum\enit@type=\tw@
	{\descriptionlabel{#1}}
	\hspace{\labelsep}%
	\else
	\ifnum\enit@type=\z@
	\refstepcounter{\@listctr}\fi
	\quad\@itemlabel\hspace{\labelsep}%
	\fi}
\makeatother
\parindent=0pt


\renewcommand{\coursedate}{осень 2022}
\renewcommand{\term}{I}
\usepackage{amsmath}
\usepackage{mathtools}
\usepackage{enumitem}
\usepackage[inline]{enumitem} 


\begin{document}
		{\heading{1. Матрицы}}
	
	\begin{problem}{
		Вычислить в зависимости от $N$: \\
		\centerline{
			\begin{enumerate*}			
			\item \(
			\begin{pmatrix}
				1 & 0 & 0 & 0 \\
				0 & 1 & 0 & 0 \\
				0 & 0 & 1 & 0 \\
				0 & 0 & 0 & 1 \\
			\end{pmatrix}^N
			\)
			
			\item \(
			\begin{pmatrix}
				1 & 0 & 0 & 0 \\
				0 & 2 & 0 & 0 \\
				0 & 0 & 3 & 0 \\
				0 & 0 & 0 & 4 \\
			\end{pmatrix}^N
			\)
			
			\item \(
			\begin{pmatrix}
				0 & 1 & 0 & 0 \\
				-1 & 0 & 0 & 0 \\
				0 & 0 & 0 & 1 \\
				0 & 0 & -1 & 0 \\
			\end{pmatrix}^N
			\)
			\end{enumerate*}
		\begin{enumerate*}[resume]
			
			\item \(
			\begin{pmatrix}
				0 & 1 & 0 & 0 \\
				0 & 0 & 1 & 0 \\
				0 & 0 & 0 & 1 \\
				0 & 0 & 0 & 0 \\
			\end{pmatrix}^N
			\)
			
			\item \begin{math}
			{\begin{pmatrix}
				1 & 1 & 0 & 0 \\
				0 & 1 & 1 & 0 \\
				0 & 0 & 1 & 1 \\
				0 & 0 & 0 & 1 \\
			\end{pmatrix}^N}\end{math}
		
		\end{enumerate*}}
	}
	\end{problem}
	\begin{problem}{	
		Докажите, что существует \textbf{единственная} матрица $E$ такая, что:
			$$\forall A \ \ EA = A = AE$$
	}\end{problem}
	\begin{problem}{	
			А верно ли следующее? ($\exists!$ означает "существует единственный")
			$$\forall A \ \ \exists! E \ \ EA = A = AE$$
	}\end{problem}

	\begin{problem}{[K 18.1а] Решить систему уравнений:
			$$\begin{dcases}
				$$X + Y \hfill = \begin{pmatrix}
					1 & 1 \\
					0 & 1
				\end{pmatrix}$$ \\
				$$2X + 3Y \hfill = \begin{pmatrix}
					1 & 0 \\
					0 & 1
				\end{pmatrix}$$
			\end{dcases}$$
	}\end{problem}
	
	\begin{problem}{	
			Найти парочку матриц с действительными элементами, удовлетворяющих уравнению:
				$$X^2 = -\begin{pmatrix}
					1 & 0 \\
					0 & 1
				\end{pmatrix}$$
	}\end{problem}
	
	\textbf{Откуда вообще взялось такое странное правило умножения матриц? На этот вопрос мы дадим ответ ближе к концу семестра, а пока посмотрим, в каких задачах его можно применить}
	
	\begin{problem}
		В стране $N$ городов, некоторые из которых соединены дорогами (список пар соединённых городов дан). Предложите способ "быстро" посчитать количество путей из города $1$ в город $N$ таких, что:
		\begin{itemize} {
			\item между 1 и $N$ было посещено ровно $K$ городов (необязательно различных или отличных от $1$ и $N$\footnote{Т.е. если мы хотим доехать из города 1 в город 3, суммарно посетив 5 городов, то вполне годится путь 13213 (если есть все требуемые дороги)}.
			\item между $1$ и $N$ было посещено не более $K$ городов (с той же оговоркой).
		}\end{itemize}
		\textit{Работают ли предложенные алгоритмы в случае, если между какими-то двумя городами есть несколько дорог? А если дороги односторонние?}
	\end{problem}

	\begin{problem}{
		Посчитайте количество нулей в матрице: \\
			\centerline{
				$\underset{(100 \times 100)}{\begin{pmatrix}
					1 & 1 & 0 & \cdots & 0 & 0 & 0 \\
					0 & 1 & 1 & \cdots & 0 & 0 & 0 \\
					0 & 0 & 1 & \cdots & 0 & 0 & 0 \\
					\vdots & \vdots & \vdots  & \ddots & \vdots & \vdots & \vdots \\
					0 & 0 & 0 & \cdots & 1 & 1 & 0 \\
					0 & 0 & 0 & \cdots & 0 & 1 & 1 \\
					1 & 0 & 0 & \cdots & 0 & 0 & 1  \\
				\end{pmatrix}^{99}}$
			}
		\textit{Как можно было бы посчитать элемент с координатами $i, j$ в этой матрице?}
		
	}\end{problem}

	\begin{problem}
		Предложите способ посчитать $N$-ое число Фибоначчи за $O(\log N)$. 	
	\end{problem}

	\begin{problem}
		Предложите способ найти $a_n$, где $a_0 = 1, a_1 = 2, a_n = 3a_{n-1} - 2a_{n-2}$, за $\mathbf{O(1)}$ \textbf{при условии, что битовый сдвиг работает за} $\mathbf{O(1)}$.
		\\
		\textit{Эту задачу вы ещё решите на ОКТЧ в этом семестре.}
	\end{problem}

\end{document}